\documentclass[12pt, oneside, a4paper]{article}
\usepackage{graphicx}
\usepackage{ifpdf}
\usepackage{amsfonts}
\usepackage[utf8]{inputenc}
\usepackage{hyperref}
\begin{document}
\title{LocusEqu Project. Final Report.}
\author{Sergio Arbeo}\date{}\maketitle

\section{Brief introduction.}

In my humble opinion, LocusEqu has achieved some pretty impressive
results in its current state. However, its limitations are more important than 
its current features.\\

In this paper, implementation, features and limitations will be
discussed, as well as some guidelines for overcoming some of the
limitations.

\section{Implementation.}

LocusEqu classes can be found under geogebra.kernel.locusequ and
geogebra.kernel.locusequ.arith packages.\\

LocusEqu distinguish two types of AlgoElements and creates two
different kinds of objects. Some AlgoElements are considered
EquationElements, like AlgoJoinPoints, and other are considered
EquationRestrictions, like AlgoPointOnPath. The main difference is
that EquationElements does not creates any restrictions, while
EquationRestriction does.

\subsection{Scope}

One of the main pillars of LocusEqu is EquationScope.This class works
just as a variable scope in any programming language but way more
simple.\\

It contains two different scopes, one for points, relating GeoPoints
to EquationPoints; and another one relating GeoElements to
EquationElements. Both scopes work as maps and can create a proper
value given the key. The algorithm for creating a new EquationPoint
relies on a simple if with a few conditions, only one of them a bit
complicated. On the other hand, the algorithm for creating new
EquationElements is a bit more complicated since it uses a parser, for
lack of a better term, explained in the next subsection.

\subsection{Parsers}
\label{ss:Parsers}

There are two parsers in LocusEqu: EquationParser and
EquationRestrictionParser. Both share the design, so only the latter
will be explained.\\

EquationRestrictionParser contains a static Map from String to
EquationRestrictionBuilder. EquationRestrictionBuilder is an interface
containing only one method that returns an EquationRestriction from an
AlgoElement and an EquationScope. Keys are obtained from AlgoElements'
getClassName method.\\

Those AlgoElements considered EquationElements retrieve the
EquationNullRestrictionBuilder only instance. This object retrieves an
EquationNullRestriction, which only generates empty EquationList.

On the other hand, those AlgoElements considered to be
EquationRestrictions retrieve an anonymous EquationBuilder that
retrieves a proper EquationRestriction instance.

\end{document}
